\documentclass[a4paper,12pt,legno]{article}
\renewcommand{\baselinestretch}{1.5}
\usepackage{t1enc}
\usepackage[longnamesfirst, round]{natbib}  % Bindet den natbib-standard fuer das Zitieren ein
\usepackage{epsfig}
\usepackage[latin1]{inputenc}   % Ermoeglicht Sonderzeichen direkt einzugeben
\usepackage[T1]{fontenc}        % Garantiert saubere Worttrennung bei Umlauten etc.
\usepackage{color}              % Farbpaket
\usepackage{amsmath,amsfonts,amssymb}   % ermoeglicht mathematische Sonderzeichen
\usepackage{ngerman}           % neue deutsche Rechtschreibung
\usepackage[english]{babel}     %
\usepackage{ae}                 %
\usepackage{graphicx}           % Ermoeglicht das Einbinden von Bildern in allen Formaten
\usepackage{longtable}          % zum erstellen von Tabellen ber mehrere Seiten
\usepackage{multirow}           % zum Verbinden von Zeilen innerhalb einer Tabelle
\usepackage{siunitx}
%\usepackage{mathptmx}
\usepackage{booktabs, caption}			% für tablegenerator
\usepackage{threeparttable, booktabs} 
\usepackage{setspace}
%\usepackage{pictexwd}           % PicTex, ein Graphikpaket
%\usepackage{pst-all, multido}   % psTricks, ein Graphikpaket
\usepackage{url}
\usepackage{geometry}


% ________________ EINRICHTEN DES DOKUMENTS ______________________%

%\bibliographystyle{plainnat_my_version}    % legt den Stil fuer das Inhaltsverzeichnis fest

\geometry{verbose,a4paper,tmargin=20mm,bmargin=20mm,lmargin=30mm,rmargin=20mm}
%\oddsidemargin 0.1in \evensidemargin 0.1in \textwidth 15.5cm \topmargin -0.4in \textheight 24.5cm
%\parindent 0cm      % legt die Seitenraender fest

\pagestyle{plain}          % leere Kopfzeile, Seitennummer in der Mitte der Fusszeile

%\footnotesize{10p}
\newcommand{\bs}{\boldsymbol}  % shortcut zur Erzeugung von fetten Sympolen in der Mathe-Umgebung

% paragraph
\newcommand{\myparagraph}[1]{\paragraph{#1}\mbox{}\\}

\begin{document}

% ________________ TITELSEITE ______________________%


%\pagenumbering{roman}   % roemische Zahlen zur Seitennumerierung

\begin{titlepage}       % Umgebung fuer Titelseite, frei gestaltbar

\thispagestyle{empty}   % keine Numerierung auf Titelseite


\begin{center}
\vspace*{0.5cm}
{\bf  \Large The Distribution of Wealth \\in a Life-Cycle Model with Durables} \\
\vspace*{5cm} 
Master Thesis Presented to the \\ Department of Economics at the\\ Rheinische Friedrich-Wilhelms-Universit\"at Bonn
\\
\vspace*{1.5cm} 
In Partial Fulfillment of the Requirements for the Degree of \\ Master of Science (M.Sc.)\\
\vspace*{8cm} 
Supervisor: Professor Thomas Hintermaier  
\vfill
Submitted in November 2017 by:\\
Eric Lustenberger \\
Matriculation Number: 2849851
\end{center}





% 
% \begin{center}
% $ $			% oeffnet und schliesst eine Matheumgebung (Trick, um den Titel nach unten zu rutschen
% \vspace{4cm}
% 
% {\LARGE TITEL}
% \vskip 4cm
% 
% Diese Seite ist frei gestaltbar
% \end{center}

\end{titlepage}

\newpage                % erzwingt an dieser Stelle einen Seitenumbruch



% ________________ INHALTSVERZEICHNIS ______________________%

\thispagestyle{empty}
 \tableofcontents   %fuegt Automatisch ein Inhaltsverzeichnis ein
\thispagestyle{empty}
 \newpage
% 
\thispagestyle{empty}
\listoftables

 \newpage
\thispagestyle{empty}
\listoffigures

 \newpage
 
	\thispagestyle{empty}
	\section*{List of Abbreviation}
	
	% using a standard table to list abbreviations, alphabetical order has to be made manually
	\begin{table}[!htbp]
		\centering
	  %\caption{Add caption}
	\begin{flushleft}  
	   \begin{tabular}{ll}
	    CEX	  & Consumer Expenditure Survey \\ \\
	    LTV   & Loan-to-Value Ratio \\ \\
	    SCF	  & Survey of Consumer Finances \\ \\
	    SIM   & Standard Incomplete Markets Model \\ \\
	    SMM   & Simulated Method of Moments \\ \\
	    US	  & United States
	    \end{tabular}%
	\end{flushleft}
	\end{table}%
	\newpage

% ________________ HAUPTTEIL ______________________%


\pagenumbering{arabic}      % Seitenzahlen wieder arabisch numerieren
\setcounter{page}{1}        % Ruecksetzen des Seitenzahlzaehlers auf 1


\section{Introduction}
\label{Introduction}

A big majority of wealth is owned by very few. Moreover, there is a substantial widening of wealth disparities since the 1970s in the US.\footnote{See for example \citet{kuhn2017income}} Causes and implications of these facts are rigorously debated in both academics and politics. In order to promote a more vigorous discussion of the wealth inequality and its evolution over time, it is important to understand which models can be applied to study the issue at hand and to what extend they may explain certain relations observed in the data.\\

The main goal of my thesis is to contribute to the net-worth literature by investigating to what extend a life-cycle model incorporating durables can match features of the net-worth cross-section in the US. While it is quite well established within the net-worth literature, which features of the net-worth distribution can be accounted for by a plausible parametrised version of the standard incomplete market model (SIM), most of these models abstract from durable goods. As \cite{FV&K2011}  indicate, this abstraction may not be well founded. The authors show that modeling durables is important to reproduce the non-durable consumption life-cycle profile observed in the data. Moreover, they point out, that durables help to explain why households with higher life-cycle income save proportionally more than poor households.\\
The model in question is a life-cycle version of the incomplete markes model with durables in \cite{hintermaier2010} with life-cycle parametrisations based on \cite{hintermaier2011}. In such a model, the restricted asset-choice set available to consumers of different ages experiencing different histories of labor income shocks, implies an endogenous wealth distribution across these consumers. \\
I calibrate the model to match empirical moments of aggregates in the Survey of Consumer Finances (SCF) 2004 data. I then compare the simulated model output to the data sample from \cite{hintermaier2011}. Their data includes three wealth distributions, each for a different age group and for the working population aged 26 to 55. For the same reasons as these authors and further documented below, I abstract from the top 10\% richest. The use of their detailed micro-level consumer data is well suited to analyze the model's performance, as it also allows for an accurate illustration of differences across the age groups concerning the net-worth distribution.\\
I find that the model accurately predicts the wealth distributions of the 26 to 45 years old and captures certain features of the 46 to 55 years old. Moreover, using data from the literature, I am able to show that the model is able to reproduce the evolution of the average portfolio and average consumption over the life-cycle and does match the relative ranking of the inequalities found in the data. 
\\ 
In addition to these main findings I conduct a counterfactual experiment in order to illustrate the importance of the loan-to-value ratio (LTV) within the model. I set the loan-to-value ratio to zero, hence durables do not have collateral value in this setting. I am able to illustrate that the level of the LTV does mostly affect the poorest consumers with high income. As younger consumers tend to hold less wealth than their older counterparts, they are most affected. Finally, since the sample from \cite{hintermaier2011} only contains initial conditions in net-worth for the simulation, I also look at conditions, which are divisible in durables and liquid assets. I then re-estimated the model for different specifications of the initial conditions finding that their specification is important for the model's prediction of the distribution of the youngest age group. 
\\ \\
Section 2 reviews the theoretical literature. Section 3 presents important empirical facts of the wealth distribution. The fourth  explains the model followed by a discussion of the calibration in Section 5 and a review of the numerical implementation in Section 6. I then turn to the model prediction starting with the discussion of life-cycle averages in Section 7. Section 8 displays the main results concerning the net-worth distribution and Section 9 discusses implications of the counterfactual experiment. Finally, Section 10 concludes. 


\section{Modeling the Wealth Distribution}
This section gives an overview of the net-worth literature. I thereby start out with a brief review of the standard incomplete market model, which is the building block of the model presented further below. I then explain, why the life-cycle version is interesting in the context of the net-worth distribution. This discussion is followed by an overview of the more recent net-worth literature. Finally, I conclude this section with a review of the literature discussing durables in this context. 

\label{Modeling the wealth distribution}
\subsection{The Standard Incomplete Markets Model}
The modeling of distributions requires some form of heterogeneity. The most common model used to achieve differences across agents is the SIM, based on the works of \cite{bewley1977permanent}, \cite{aiyagari1994}, \cite{huggett1993risk}.\footnote{See \cite{ljungqvist2012recursive} for a more technical review.} In this model, ex-ante identical consumers are subject to uninsurable idiosyncratic income risks. In order to insure themselves against these income shocks, agents accumulate precautionary savings. Since income histories vary according to the agent, the latter accumulate different amounts of savings resulting in an endogenous wealth distribution. The driving feature is thereby the exogenously determined imperfect market structure. The market imperfection is implied by a limited asset choice-set for consumers \--- In a complete market consumers could achieve perfect insurance by trading contingent claims and would thus not accumulate precautionary savings. The canonical version allows for a one-time period bond only, as for example in \cite{hintermaier2011}. Including durables enlarges the choice-set for consumers and allows for the study of the portfolio choice. Moreover, the expansion of the choice-set may also affect net-worth, since consumers are more flexible in their ability to selfinsure against idiosyncratic risk.  

\subsection{The Life-Cycle Feature}
One important extension to the early, infinite-horizon version of the SIM was the life-cycle model (see \cite{huggett1996wealth} for an early application).\\
The choice of a life-cycle model, which produces a net-wealth distribution arising from rational choices of consumers subject to income uncertainty, allows for a detailed analysis of consumption and savings behavior over a person's life-span. 
Explicitly modeling an agent's life-cycle makes room for forms of heterogeneity across ages and thus the inclusion for further determinants of savings. This is particularly interesting for the present analysis, since, as I will discuss in Section \ref{facts}, there are strong heterogeneities across age-groups in the distribution of wealth as well as in the portfolio composition in the data.\\
One important feature concerns the modeling of a retirement period, introducing life-cycle risk to precautionary savings i.e. consumers need to accumulate enough savings during their active working life in order to smooth consumption across their whole life-span.\footnote{See \cite{cagetti2003} or \cite{Gourinchas&Parker2002} for a discussion of the relative importance of these saving determinants across the life-cycle.} Similarly, one may model deterministic income growth for the active working population. Both are changes in expected future income and thus affect the savings behavior today. 
\\
The more recent literature on determinants of the net-worth distribution, discussed in the next section, almost unanimously employs versions of this type of models and enriches the standard version with various heterogeneities across the life-cycle in order to improve its prediction of the net-worth distribution.

\subsection{Wealth Distribution Literature}
The endogenously generated distribution of wealth within the model makes the SIM a natural candidate to study the wealth-distribution.
Following \cite{huggett1996wealth} many authors have implemented its life-cycle version to analyze how well the simulated model distributions match the data. Thereby, one particular feature of the wealth distribution has been dominating the research in this field: The modeling of the extreme concentration of wealth at the right tail of the distribution. While many features of the net-worth distribution can be accounted for in more canonical versions of the SIM, they perform very badly when it comes to the very rich.\footnote{See \cite{huggett1996wealth} or \cite{quadrini1997understanding} for an illustration.} \\
\cite{castaneda2003} have proposed to model extreme earning shocks for a very small part of the population. The authors show that their model can account for the earnings and wealth inequality observed in the US in a model including retirement, altruism, government transfers to the retired and earnings risk (see for example \cite{diaz2010} or \cite{kaymak2016evolution}, for more recent implementations of this approach). \cite{hintermaier2011} decide to abstract from the top 10\% of the distribution and show that, a more parsimonous version of the model with a more realistic representation of the income process can predict some parts of the evolution of the net-worth distribution in the US between the 1980s and the 2000s.\footnote{The issue with the approach implemented by \citet{castaneda2003} is that their income process is calibrated based on matching the Lorenz curves of both earnings and wealth inequality and hence, is not estimated based on micro-level earnings data. As a consequence it remains unclear, whether the actual earnings process can produce the strong concentration of wealth at the right tail of the distribution.} \cite{krusell1998} were the first to point out that a stochastic discount factor across dynasties can account for the variance of the cross-sectional distribution of wealth and does increase the wealth concentration among the richest. Other approaches are: A heterogeneous rates of return \citep{benhabib2011distribution}, a richer earnings process \citep{denardi2016}  or the role of entrepreneurship \citep{cagetti2009}. \cite{denardi2014} show that adding voluntary bequests and intergenerational transmission of earnings drastically improves the model's prediction for the empirical cross-sectional differences in wealth at retirement as well as their correlation with lifetime incomes.  Although these additional modeling features are able to better approximate the right tail of the wealth distribution, they still fail to produce a satisfying representation.\footnote{For a more indebted review and the shortcomings of the different modeling approaches of this literature, see \cite{denardi2017}.}  \\
Whereas the literature demonstrating that a plausibly parameterized version of the SIM can quantitatively explain features of the net-worth distribution is quite large, few of these models incorporate durables.

\subsection{Durables and the Net-Worth Literature}

Modeling durables is particularly interesting since they exhibit a dual role. On the one hand, consumers can derive utility services from their durable stock, on the other hand, due to their durability, these types of goods may be used as collateral. Modeling durables thus often entails an endogenous borrowing constraint, where depending on the level of their durable stock, households may borrow more or less. \\
\cite{yang2009} shows that borrowing constraints are important to explain the accumulation of housing assets early in life. \cite{FV&K2011} stress the importance of the dual role of durables in the accumulation of assets. They show that in a model with durables the dispersion in wealth-accumulation between households with high- and low life-cycle income rises compared to a model abstracting from durable goods. Moreover, they underline the role of durables to generate the life-cycle profile of consumption observed in the data. Further, \cite{gruber2003precautionary} show that durables are important for the accumulation of precautionary savings.    
\\
Additionally, including durables enables for a discussion of portfolio composition \--- Either across the life-cycle as in \cite{yang2009} and \cite{FV&K2011} or across the wealth distribution as in \cite{diaz2010}. The latter use an infinite-horizon model with housing to look at empirical predictions of the wealth distribution and find that their model is quite successful in replicating the portfolio composition across the net-worth distribution.
\\
There is thus substantial support within the theoretical literature to model durables. Moreover, durables make up a large part of the fraction of wealth held by the households. 

\section{Facts about the Net-Worth Distribution} \label{facts} There are two main points in the data, which I wish to highlight. Firstly, there are non-negligible heterogeneities across different age groups in terms of within-group inequality and the mean holdings of net-worth. Secondly, durables are an important part of the households' net-worth portfolio. This share is particularly important for young consumers and up to the 90th percentile of the net-worth distribution. 


\subsubsection{Net-Worth and Changes across Age-Groups}
The results in this subsection are based on calculations by \cite{hintermaier2011} on the cross-section of the SCF 2004. Net-worth is measured in units of average population net labor- earnings, which is adjusted by the household size using the equivalent scale introduced by \cite{fernandez2007consumption}. \\
There are substantial differences in the net-worth distribution across different age groups in the 2004 US data. 
Table \ref{facts_changes} displays the gini-coefficients and the means for three different age groups, both for the whole sample, as well as up to the 90th percentile.\\ 

\begin{table}[!htbp]
\centering
\caption{Facts: Means and Ginis of the Net-Worth Distribution}
\label{facts_changes}
\begin{tabular}{@{}llll@{}}
\toprule
                     & Age 26-35 & Age 36-45 & Age 46-55 \\ \midrule
\textbf{Full sample}          &           &           &           \\
\ \ Mean                 & \ \ \ \ 2.28      & \ \ \ \ 5.90      & \ \ \ \ 12.49     \\
\ \ Gini coefficient     & \ \ \ 0.824     & \ \ \ 0.765     & \ \ \ \ 0.765     \\ \\
\textbf{Sample $\leq$ 90th pctile} &           &           &           \\
\ \ Mean                 & \ \ \ \ 0.80      & \ \ \ \ 2.36      & \ \ \ \ \ 4.81      \\
\ \ Gini coefficient     & \ \ \ 0.713     & \ \ \ 0.596     & \ \ \ \ 0.564     \\ \bottomrule
\multicolumn{4}{l}{%
  \begin{minipage}{11.5cm}%
    \small Ginis coefficients and average net-worth holdings by age group for the full net-worth distribution and up to the 90th percentile of the net-worth distribution of 2004 U.S data. The unit of measurement of the means is the average net labor-earnings (adjusted for household size). Source: \cite{hintermaier2011} Table 2. Based on the calculations of these authors using 2004 SCF data. 
  \end{minipage}%
}\\
\end{tabular}
\end{table}

The average wealth holding is more than $2$ times larger with each age group, for both the full distribution and the distribution up to the 90th percentile. Mean wealth thus increases substantially across age groups. While average wealth is increasing, the gini-index indicates that the within-age group inequality tends to be decreasing. The only exception is the stagnation between the older age groups for the full distribution. \cite{hintermaier2011} show that this pattern holds true for eight surveys from 1983 to 2007, with the exception that inequality tends to  increase between the last two age-groups for the whole sample.\footnote{See Table 2 in their paper.} \\
Furthermore, the differences between the full sample and the 90th percentile sample indicate that the wealth held by the top 10\% is quite significant and its magnitude is relatively increasing in the age groups. Figure \ref{scf_data} confirms these findings. It plots the detailed net-worth distribution for different age groups of the US in 2004. The solid line describes the distribution up to the 90th percentile. The dotted line indicates the distribution for the top 10\%. 
\begin{figure}[!htbp]
\caption{Net-Worth Distribution per Age Group, SCF 2004 Data} 
\label{scf_data}	%label, um spaeter auf die Graphiknummer zugreifen zu koennen
\centering
\includegraphics[scale=.4]{Figures2/data_plot}  % width legt Breite der Graphik fest

\begin{minipage}{0.8\linewidth}
\footnotesize{Figure \ref{scf_data} displays the empirical distributions for the age groups, 26-35, 36-45 and 46-55. The solid line represents the distribution up to the 90th percentile and the dashed line the top 10\% of the distribution. The unit of measurement is the average of net labor-earnings (adjusted for household size). Source: \cite{hintermaier2011}. Based on the calculations of these authors using 2004 SCF data. }
\end{minipage}

\end{figure}

\subsubsection{Portfolio Composition and the Importance of Durables}
Two things are particularly noteworthy about durables. Firstly, \cite{kuhn2017income} show that households in the bottom 90\% hold almost their entire wealth in durables and hence are not diversified in their assets.\footnote{These authors analyze long-term trends in the distribution of US household income and wealth over the past seven decades introducing a newly compiled household-level dataset based on the SCF.} This picture changes drastically for the top 10\%, who instead hold a large share in business equity and other financial assets. Secondly, such heterogeneities also exist across the life-cycle. 

\cite{FV&K2011} use the SCF 1995 to document aspects about the life-cycle profile of household assets. Before consumers reach the age of 40, housing is more important than total wealth (sum of housing and financial wealth) of consumers holding at least some assets. Only afterwards is the fraction of housing lower than total wealth, however this fraction always stays above 50\%. \cite{yang2009} finds a similar pattern for  homeowners using six waves of the SCF (1983-1998) data. At a young age,  homeowners hold slight negative average financial asset positions and a reasonable average stock of housing. Both stocks increase, however the financial stock overtakes housing in the mid-40s and then continues to grow at a fast-increasing rate up to the age of 70, while the housing-stock reaches its peak at the age of 55 and then flattens out.\footnote{The average asset holdings of renters are also increasing over the life-cycle, however at a much lower rate and initially positive.} The young thus tend to hold most of their wealth in durables, and only start accumulating financial wealth later on in life.
\\ These facts show, that explicitly modeling durables seems particularly interesting for the sample of 26 to 55 years old up to the 90th percentile.

\section{The Model}
\label{model}
In the following section I will present the life-cycle imperfect market model with durables.\footnote{The model is a life-cycle version from the one presented in \cite{hintermaier2010}.} After describing the consumer's problem and the income process, I will give an overview  of the savings' determinants accommodated by the model, which give rise to the endogenous distribution and are therefore at the center of discussion, when assessing the fit of the model with the data.

\subsection{Demographics}
There is a continuum of ex ante identical risk averse consumers with a finite time horizon. At age 90 consumers die with certainty, younger consumers between the age of 26 and 90 may die with probability $\pi_{j} < 1$ at age $j$.
\subsection{Consumer's Problem} 
\myparagraph{Consumption savings decision}
Consumers derive utility from a durable good $d$ and a non-durable good $c$. The utility function $U(c,d)$ is strictly increasing in both arguments and strictly concave, with diminishing returns for both $c$ and $d$. Moreover, the utility function allows for non-separability in $c$ and $d$. The choice-set of assets is restricted to durables $d$ and a one-period bond $a$. Preferences are assumed to be time separable and future utility is discounted with a constant discount factor $\beta$.
In each period, after receiving the income $y_{ij}$, consumers decide on the amount they want to consume in this period $c_{j}$ and next periods portfolio composition $d_{j+1}$ and $a_{j+1}$. Agents then derive utility from consumption and durable services, before the assets pay returns. While the liquid assets return an interest rate $r$, durables depreciate at rate $\delta$. This implies the following budget constraint:  

\begin{equation}
a_{j+1}+d_{j+1}+c_{j}=(1+r)a_{j} + (1-\delta)d_{j}+y_{j}
\end{equation}

\myparagraph{Income Process}
I directly follow the specification of the income process in \cite{hintermaier2011}. Until retirement every consumer $i$ receives an individual, stochastic labor income $y_{ij}$ in each period. After the age 65 is completed, they retire with certainty and receive deterministic individual-specific retirement benefits $b_{i}$. \\
The authors assume that log of earnings $y_{ij}$ of individual $i$ at age $j$ before retirement is additively separable in a deterministic age polynomial $\phi_{j}$ and an idiosyncratic income shock $z_{ij}$:\footnote{There is some evidence that this process is too simplified. \cite{guvenen2015data} proposes a more complex income process, which provides a better fit to the data. However, I decide to stick with the simpler as the main aim is to establish the fit of a model with durables.} 

\begin{equation}\label{eq:income_process}
y_{ij}=\phi_{j}+z_{ij}
\end{equation},


where the shock $z_{ij}$ follows an AR(1) process

\begin{equation}\label{eq:ar1_shock}
z_{ij} = \rho z_{i,j-1}+\epsilon_{ij}
\end{equation}


\myparagraph{Collateral Constraint}
Durables serve both as consumption good and value storing asset, since their durability implies that they may be used as collateral for short-sales of the liquid asset. It is assumed that all credit needs to be collateralized, either by income or by durable holdings.\footnote{As \cite{hintermaier2010} point out, this assumption is reasonable since about 85\% of household debt in the SCF 2004 is secured by collateral. However, as I show below, this assumption may have some repercussions.} The collateral constraint takes the form:

\begin{equation}\label{eq:borrowing_constraint}
\underbrace{\mu(1-\delta)d_{i,j+1} + \gamma\underline{y}}_{collateral} \geq -(1+r)a_{i,j+1}
\end{equation}

where $\underline{y}$ is the minimum labor income realization across all states and all periods and $\mu \in [0,1) and \gamma \in [0,1)$ are the respective fractions of the durable stock and of minimum labor income, which can be collateralized. The timing assumptions made above imply that this constraint guarantees full repayment by consumers and accordingly acts as non-bankruptcy constraint.\footnote{This is assured by the fact, that the lender does know the financial portfolio choice $(a_{j+1},d_{j+1})$ and the minimum of the support of the income distribution across all ages and periods.} A possible interpretation for $\gamma\underline{y}$ is that it consist of all debts where the wage plays a major part in the risk assessment from the lender's side i.e. credit card debts or short-term unsecured loans. 


\myparagraph{Net-Worth in the Model} 
Net-worth is defined as:
\begin{equation}\label{eq:net_worth}
x_{i,j} \equiv (1+r)a_{,ij} + (1-\delta)d_{i,j},
\end{equation}
And thus reflects the total amount of assets a consumer has access to in a given period.

\subsection{The Recursive Formulation of the Household Problem} 
\myparagraph{Prior to Retirement}
We let $T^{r}$ denote the first period of retirement.
The Bellman equation prior to retirement, if $j < T^{r}$ thus is:
\begin{equation}
v_{j}(x_{j},d_{i,j},y_{i,j}) = \max_{\substack{a_{i,j+1},d_{i,j+1}}}\left[U(\underbrace{x_{i,j}+y_{i,j}-a_{i,j+1}-d_{i,j+1}}_{c_{i,j}},d_{i,j})+\hat{v}_{j}(x_{i,j+1},d_{i,j+1},y_{i,j})\right]
\end{equation}

where the expected next period value function is discounted by the product of the probability of survival $(1-\pi_{j})$ and the discount factor $\beta$
\begin{equation}
\hat{v}_{j}(x_{i,j+1},d_{i,j+1},y_{i,j}) \equiv \beta (1-\pi_{j})E_{j}v_{j+1}(x_{i,j+1},d_{i,j+1},y_{i,j+1}),\textnormal{\footnote{Note that the income $y_{i,j}$ ensters the expected value of next period's utility function as a state, due to its persistence.}}
\end{equation}

with the constraints: 

\begin{equation}
a_{i,j+1}+d_{i,j+1}+c_{i,j}=x_{i,j}+y_{i,j},
\end{equation}

\begin{equation}
x_{i,j+1} = (1+r)a_{i,j+1} + (1-\delta)d_{i,j+1},
\end{equation}

whereby the borrowing constraint is rewritten in terms of future durable and future net-worth holdings (using Equations \ref{eq:borrowing_constraint} and \ref{eq:net_worth}): 

\begin{equation}\label{eq:borrowing_constr_net_worth}
x_{i,j+1} \geq -\gamma\underline{y}+(1-\mu)(1-\delta)d_{j+1}, 
\end{equation} 

\begin{equation}
d_{i,j+1} \geq d_{min},
\end{equation}

\myparagraph{Post Retirement}
After retirement, for periods $j \geq T^{r}$, the income $y_{ij}$ is given by $b(z_{i,T^{r}})$, the retirement benefits $b_{i}$ depending on the last realization of labor income before retirement $z_{i,T^{r}}$.

\subsection{Equilibrium Concept}
The interest rate is fixed, assuming that changes in the domestic supply of assets do not affect the economy. This assumption corresponds to a small open economy, where prices are determined by the world market. Moreover, as the price is observed in the ex-post analysis it suffices to determine the equilibrium asset quantities \citep{hintermaier2011}. Furthermore, it is assumed that the assets of agents, who die are taxed away and that equilibrium feedbacks from the government budget constraint are negligible. 


\subsection{The Determinants of Savings}
\label{determinants}
The model accommodates several determinants of savings, the accumulation of net-worth, which I will briefly outline here. These will be further discussed in the subsequent analysis. 


\myparagraph{Income Risk} Agents accumulate buffer stocks of savings in order to self-insure against labor income uncertainty and being able to smooth consumption across different states of the income process. The risk is captured by the AR(1) process specified by Equation \ref{eq:ar1_shock} and thus depends on the distribution of the error term of the process in question as well as the persistence factor. \\
\cite{aiyagari1994} quantifies the aggregate importance of buffer stock savings. \cite{Gourinchas&Parker2002} assess the relative importance of these precautionary savings across the life-cycle of a consumer. They show that the precautionary motive dominates the accumulation of savings early in life, while later in life, after the age of 40, savings, due to deterministic changes in income, become more important. 

\myparagraph{Income Growth} In the model above there are two sources for deterministic changes in income growth, which give rise to life-cycle savings motives. Firstly, for the working population it is the experience premium, the deterministic age polynomial in Equation \ref{eq:income_process}. Secondly, every household retires at 65 and experiences a large drop in income. These factors together make up for a hump shaped income profile displayed in Figure \ref{consumption_life_cycle} in Section \ref{life_cycle_profiles}. The characteristic of these changes is, that they are perfectly anticipated by the consumers. An expected higher income causes perfectly rational consumers to reduce their present savings and increase consumption as they seek to smooth consumption across their life-cycle. On the contrary, when facing a large drop, such as retirement, consumers increase present saving. 

\myparagraph{Borrowing Constraint} Another important determinant of savings in these models is the borrowing constraint. A tighter constraint limits the insurance potential of assets. They limit agents in their ability to short-sell assets and thus trade-off future consumption against present consumption. \\ 
The borrowing constraint from Equation \ref{eq:borrowing_constr_net_worth} implies that in each period net-worth needs to be larger than the negative value of the minimum fraction of income that can be collateralized and the fraction of wealth which cannot be used as collateral. As I will show further below, the calibration implies that although allowing for negative values of net-worth, these are negligibly small and thus similar to \cite{hintermaier2011}, who estimate the borrowing limit to be zero for the observed distribution of net-worth in the present paper. 
\\ 
As introduced above, an important distinction between a more standard specification of the borrowing constraint and the present constraint, is that the present collateral constraint is endogenous due to the collateral value of durables. Although the limit for net-worth may be similar, the collateral value of durables gives agents more freedom to smooth consumption. As I will discuss futher below this has implications for the accumulation of net-worth early in life.\\ The role of durables as collateral value further implies that the LTV is an additional determinant of savings. A higher LTV relaxes the constraint and thus allows for more borrowing.\footnote{\cite{cho2012accounting} shows that changes in the LTV can account for 40\% of the difference in homeownership between Korea and the US.} I discuss the LTV extensively in Section \ref{counterfactual}.
Finally, the relative price of durables does also affect savings.

\myparagraph{The Relative Price of Durables} \cite{FV&K2011} show that the dual role of durables leads to a higher dispersion in net-worth due to the former's relative price. High-income earners are able to accumulate more durables. As their durable stock increases, the marginal utility of durables falls and financial assets become relatively more attractive. Consumer with higher life-time income do therefore hold more durables and more financial assets than consumers with low income. This increases the net-worth dispersion between different levels of life-time income. 

\section{Calibration}
I carefully calibrate my model to mimic the real economy, which in this case is the US in 2004, as close as possible. Additionally the calibration strategy should be in line with the aim of the present study, such that matched features are not the ones under study. I thus chose a set of parameters from the empirical and net-worth literature and calibrate the discount factor and the non-durable consumption weight to match empirical moments from the SCF 2004 data. These preference parameters are chosen such that the model matches the means of durables and net-worth of the sample of interest observed in the data. For the calibration of the income process I directly follow \cite{hintermaier2011}, who approximate the labor income observed in the SCF 2004 data. 

\subsection{The Income Process}
\cite{hintermaier2011} use the SCF cross sections to construct a measure for labor-earnings risk before retirement purging labor-earnings from age effects for consumers between age 26 and age 65.\\ 
The age polynomial $\phi_{j}$ from Equation \ref{eq:income_process} for the 2004 SCF data between ages 26 and 65 is obtained by regressing the log of earnings on a quartic age polynomial in the respective survey year. The table in appendix \ref{table_experience_premium} displays the experience premiums obtained by \cite{hintermaier2011} for the year 2004, which demonstrates a hump shaped average income for the working age population.
The standard deviation of the residuals resulting from this regression is used to calibrate the distribution of earnings shocks $z_{ij}$. Assuming normality of the error terms the authors found a variance of $0.607$ for the 2004 data \---  $z_{2004} \sim \mathcal{N}(0,0.607)$.\footnote{\cite{guvenen2015data} finds that earnings shocks show large deviations from lognormality, displaying a strong negative skewness and extremely high kurtosis, which contradicts the normality assumption implemented here. However, as the aim of this study is not to implement an empirically more accurate earnings process, I will stick to the calibration by \cite{hintermaier2011}.} The autocorrelation of the log-earnings shocks is calibrated as $\rho = 0.95$, which implies a variance for the innovations of $\epsilon_{ij} = 0.048$. For the simulation, the AR(1) process for $z_{ij}$ is approximated by a Markov chain with 21 income states via the Rouwenhorst method. \\
After retirement the income process is deterministic. As a result, each individual receives retirement benefits from social security, the level of which is determined by the last period's income resulting in a replacement ratio of benefits over gross income of 52\% for the median income in the last period before retirement. The approximation takes into account the US social security legislation (http://www.ssa.gov).\footnote{For a detailed description of the construction of these benefits I refer directly to \cite{hintermaier2011}.} \\
Further, the authors adjust for growth in life-cycle income to convert the cross-sectional age-earnings pattern into life-cycle profiles considering a growth factor of $1.015^{(age-base age)}$, where the base age, age 20, is a reference age to make income units comparable across cohorts of different years.\\
Finally, it is important to note that this calibration implies a smallest income larger than one for all ages, which signifies that the collateralizable income $\underline{y}$ is larger than zero. The income growth and the calibration of the retirement benefits further imply that the collateralizable income corresponds to the smallest income in the first period and is well defined across all periods, i.e. in all other periods, including retirement, the minimum income is larger. 

\subsection{Utility Parameters}

The utility function considered is non-separable in durable- and non-durable consumption, obeys the Inada-Conditions with respect to nondurable consumption and fullfills the criterias discussed in Section \ref{model}:\footnote{It is the same class of preferences as in \cite{hintermaier2010} and in line with a more generic formulation of the utility function \citep{FV&K2011}.}

\begin{equation}
U(c,d)=\frac{\psi(c,d)^{1-\sigma}-1}{1-\sigma} \ \ \textnormal{where} \ \ \psi(c,d)=c^{\theta}(d+\epsilon_{d})^{1-\theta},
\end{equation}

where I assume $\epsilon_{d} > 0$, which is a number small enough to be irrelevant for the quantitative exercise at hand but nonetheless larger than zero. The CRRA utility function with the Cobb-Douglas specification of the consumption index, thus allows for zero consumption of durables, while the Inada-Condition ensures that people always consume some non-durables. Intuitively this means that consumers cannot survive without food, but are allowed to survive without houses and cars.\footnote{Since in such a formulation of the problem durables are not "naturally" bounded below by the Inada-Conditions, when solving the recursive problem one has to take into account $d' = d_{min}$ as additional constraint.}

The risk-aversion $\sigma$ typically takes values from 1-5 in the literature \citep{yang2009}. Following the aforementioned author, I use $\sigma = 1.5$ as estimated by \cite{attanasio1999} and \cite{Gourinchas&Parker2002} from consumption data. The discount factor $\beta$ and the weight on non-durable consumption $\theta$ are calibrated to match the average net-worth holdings and average durable holdings of the prime age population up to the 90th percentile, respectively. The data moments in terms of the average labor-earnings equivalent are 2.95 for the durable holdings and 2.39 for the net-worth  holdings. Since these data moments are not available in \cite{hintermaier2011} or any literature for that matter, I calculated these with Stata-Codes from \cite{hintermaier2016}. Appendix \ref{data} provides further information on this data. To calibrate the two parameters I solved the model for economical plausible combinations of $\beta \in [0.96,1]$ and $\theta \in [0.65,0.90]$ to calculate the equivalent model moments, starting with differences of 0.05 between parameter values and then reducing them to 0.01 for a smaller grid of parameter values.\footnote{Note that unlike in infinite-horizon models, the life-cycle version does not demand a $\beta < 1$ to keep life-time utility finite \cite[p. 360]{heer2004dge}.} The choice to match the 90th percentile and the population from 26 to 55 years old is motivated by the fact that calibrating the model for the whole data-set would lead to overestimating the net-worth holdings of the prime sample. Both the population
above 55 years of age and above the 90th percentile hold more wealth on average than
in the prime age sample. The matching of the moments is explained in more detail in Appendix \ref{estimation_procedure}. \\
Table \ref{estimates} displays the parameter values that best match the two moments. Moreover, it features the simulated moments, which match the empirical moments with an accuracy of $10^{-1}$.

\begin{table}[!htbp]
\centering
\caption{Calibration Results}
\label{estimates}
\begin{tabular}{@{}llll@{}}
\toprule
\ \ \ $\beta$ & \ \ $\theta$ & \ Av.Durables & \ Av.Net-Worth \\ \midrule
0.991 \ \ \ \   & 0.764 \ \ \ \ & \ \ \ \ 2.9560 & \ \ \ \ 2.3886  \\ \bottomrule
\multicolumn{4}{l}{%
  \begin{minipage}{9.5cm}%
    \small This table depicts the parameter estimates matching the average durable holdings and average net-worth of the prime age population up to the 90th percentile of the net-worth distribution found in the data and the simulated moments.  
  \end{minipage}%
}\\
\end{tabular}
\end{table}

These results are well in the range of the estimates of other authors with similar models and utility specifications. \cite{FV&K2011} estimate a weight on non-durable consumption of $0.81$, while \cite{hintermaier2016} find a weight on non-durable consumption of $0.76$ for a model with housing for the same sample and data, \cite{gruber2003precautionary} finds a value of $0.7$ for a model with housing and a similar sample.
The calibration for $\beta$ is quite close to \cite{hintermaier2011}, who estimate the discount factor to be $0.9845$ for the same sample and 1983 data. 

\subsection{Initial Conditions}
In order to properly estimate the model, the specification of the initial conditions is important in a life-cycle model. \cite{hintermaier2011} provide detailed net-worth data. However, these initial conditions do not contain any information about the portfolio composition. I thus construct the consumers' initial portfolio from the distributions of consumers between ages 23 and 25 in the SCF 2004 and correcting for average growth rates, following \cite{hintermaier2011}.\footnote{As for the data moments needed for the calibration, I use the Stata-Codes provided by \cite{hintermaier2016}. Appendix \ref{data} discusses this data in more detail.} In the following I will refer to the calibration with these initial conditions as baseline case. It is also the calibration I use for the discussion of the main results further below.\\
To test the importance of the initial conditions, I re-calibrate the model for different specifications of the initial conditions using the conditions provided by \cite{hintermaier2011}, which only indicate net-worth. In a first estimation I assign all net-worth to durables and in a second estimation I assume that the net-worth positions are entirely held in liquid assets. The results are displayed in Appendix \ref{initial_conditions} and indicate that the simulated net-worth distribution from the baseline case and the simulation assuming that 23-25 years old hold the entirety of their net-worth in durables, do produce almost identical results. This, should however not be surprising. As discussed in Section \ref{facts}, young consumers hold almost their entire wealth in durables. When assigning the net-worth to financial assets, however, the match for the empirical distribution of the 26 to 35 years old, deteriorates substantially.    


\subsection{Further Inputs}

As is common in the literature dealing with the period of the great moderation and supported by empirical evidence, the interest rate $r$ is set to $4\%$.\footnote{See for example \cite{FV&K2011} or \cite{hintermaier2011} and the reference therein.}. As discussed above, this partial equilibrium approach entails the assumption of a small open economy. Furthermore, I set the loan-to-value ratio $\mu = 0.97$, which corresponds to the legal maximum of the LTV reported in \cite{green2005american} in Table 2. Following \cite{hintermaier2010} $\gamma = 0.97$ is chosen to be smaller than $1$ in order to assure positive consumption at the smallest gridpoint of next periods' net-worth $x'$. Finally, the probabilities of death are the same as in \cite{hintermaier2011}.\footnote{They correspond to Table 1 of the decennial life table 1999-2001 published by the National Center for Health Statistics at $http://www.cdc.gov/nchs/products/life_tables.htm$.} 

\section{Numerical Algorithm}

The problem above does not provide a closed-form solution and thus the solution has to be approximated numerically. As indicated above, this model is essentially a life-cycle version of the one discussed in \cite{hintermaier2010} and therefore perfectly suitable to be solved by the solution algorithm discussed therein. In their paper, the authors expand the endogenous gridpoints method (EGM) proposed by \cite{carroll2006} for a problem with two states and thus providing a very efficient algorithm to solve models with durables and collateralized debt. As their algorithm is formulated recursively, it is well suited to solve life-cycle models. I refer to their paper for a technical discussion and focus on the implementation of the algorithm for present model.\\
I iterate over the policy function starting in period $j = T$ where the consumer sells all liquid assets and durables, since he knows that he will die with certainty and thus wants to consume everything before death.\footnote{Note that due to the timing assumption the consumer does still derive utility from the present stock of durables. He only decides run down the stock for next period in order to consume a maximum amount.} As a result $x'=d'=a'=0$ and therefore the initial consumption policy is $c(x,d_{j},y_{kj})=x+y_{j,k}$. Each iteration $n$ on the policy function then gives the solution for the period $T-n$, where in my case $n=65$ which corresponds to ages $90$ to $26$. \\
The exogenous grid of d is chosen such that the minimum of durable holdings $d_{min} = 0$.\footnote{As discussed above, the formulation of the utility function allows for minimum durable holdings.} The minimum of the x-grid then results from the collateral constraint and is equal to $-\gamma\underline{y}$. The number of grid points are 100 and 225 for the d-grid and x-grid respectively. Moreover, I choose the grids such that regions of the policy function with higher curvature, for low values of the endogenous states, contain more grid-points. The maxima of the gird are chosen such that for the number of grid-points specified the quantitative results are not affected. I simulate 100'000 agents over their life-cycle. I abstract from agents with choices outside of the grids. I find this is a reasonable approach, as the maximum amount of agents deleted from the simulated sample when calibrating the model is 17. The additional gain of an increased number of gridpoints would not trade-off the additional computational burden.\\
Finally, since income is deterministic during retirement period, the transition matrix is equal to the identity matrix for the corresponding iterations.

\section{Life-Cycle Profiles}
\label{life_cycle_profiles}

Although the model is not calibrated to match life-cycle profiles it is interesting to see how well it reproduces the observed life-cylce patterns in the data. Moreover, this is particularly interesting in the context of durables as different authors have argued the importance of modeling durables in life-cycle models to match the empirical profiles \--- non-durable consumption in the case of \cite{FV&K2011} and the portfolio composition in \cite{yang2009}. \\
In the following section I will compare the average over the simulated life-cycle profiles to the data cited in other work. I first discuss the profile of consumption and income and then take a closer look at the simulated portfolio profiles. 

\subsection{Income and Consumption}

\myparagraph{Income} Figure \ref{consumption_life_cycle} shows the hump of the income profile arising by construction as the process is determined exogenously. Note that this hump shaped profile arises from the experience premium captured by the estimated age polynomial as well as economic growth, up to age 65. Consumers earnings rise until the age of 49, when their deterministic income peaks. Afterwards the experience premium decreases, as the experience factor is dominated by a decrease in productivity to the age of 65, where consumers earn around 10\% less than when they are at their top. At this point the average income almost halves itself as every consumer retires after the age of 65 and receives deterministic retirement benefits from then on.\footnote{Following \cite{hintermaier2011} the growth adjustment is neglected for the retirement period, therefore the curve does not increase for ages 66 to 90.} 

\begin{figure}[!htbp]
\caption{Average Income and Consumption over the Life-Cycle} 
\label{consumption_life_cycle}	%label, um spaeter auf die Graphiknummer zugreifen zu koennen
\centering
\includegraphics[scale=.4]{Figures2/life_cycle_consumption_income}  % width legt Breite der Graphik fest

\begin{minipage}{0.8\linewidth}
\footnotesize{This figure displays the averages over the simulated life-cycle profiles. The red line displays average deterministic income over the life-cycle resulting form the calibration. The blue line shows the average consumption over the life-cycle.}
\end{minipage}

\end{figure}

\myparagraph{Consumption} As income consumption is hump shaped and seems to follow the behavior of income with a slight time lag and smoother decrease. Non-durable consumption peaks on average at the age of 73. At that point agents consume almost four times as much as at the age of 27, when their non-durable consumption is lowest.\footnote{Note this first fall of consumption is due to adaptation from initiating durables and net worth.} \\
Clearly, agents do not achieve perfect consumption smoothing over their life-cycle. As in \cite{FV&K2011}, the double role of durables, which provide consumption services that are non-separable from non-durable consumption and may be utilized to substitute wealth across periods, contributes to a large extend to the hump shape of consumption.\\
Durables contribute to initially low non-durable consumption in two ways. Firstly, as the initial durable stock is low, borrowing constraints are tight. As a consequence, the consumers' ability to substitute expected higher future income for lower present income is limited leading consumption to track income quite closely. Secondly, 
due to their collateral value, consumers seek to accumulate durables as quickly as possible and thus substitute a part of their non-durable consumption with durable services. \\
This initial surge of the durable stock can be observed in Figure \ref{asset_holdings_life_cycle} and has two causes. The first is that this loosens the borrowing constraint. Agents may borrow more, which facilitates consumption smoothing and non-durable consumption behaves increasingly differently from income. The second consequence is that due to the non-separability in the utility function, the marginal utility of non-durable consumption relatively increases compared to the marginal utility from durable services. As a consequence of these two aspects consumption increases substantially. 
Finally, later in life, when death becomes more likely and utility is discounted to a larger extent, consumption decreases again.\footnote{The late rise in consumption is due to life-time uncertainty. As consumers are never entirely sure at what point they will die, they hold a small buffer even when retired and subject to deterministic income. In the last period, they die with certainty and thus sell all of their assets increasing consumption.} 

\myparagraph{How Does the Consumption Profile Compare to the Literature?} \cite{FV&K2011} and \cite{yang2009} have looked at life-cycle profiles using very similar models accompagnied by an overview of the empirical profiles.\footnote{Note that these studies use the same household equivalent from \cite{fernandez2007consumption} to control for family size, when looking at the data, as \cite{hintermaier2011} and are thus suitable for comparison.} They both find similar shapes, however, differing in the size and period of the peak.\\
Using data from the Consumer Expenditure Survey (CEX) for the years 1980-2001 \cite{FV&K2011} show that the average household spends around 25\% more in their early 50s, when controlling for family size. Yang uses CEX 1984-2000 data for homeowners and reports a peak of 1,56 times larger at age 51 compared to average consumption at age 20. Both profiles display low initial consumption, a steady rise and then a fall mirroring the rise in early years. \\
Their models are able to match these findings quite accurately, reproducing a peak  at age 45, which is 40\% higher compared to the level at age 20 in the case of \cite{FV&K2011} and a peak at age 60, 90\% higher than at age 25 for home-owners in the case of \cite{yang2009}. 
\\ \\
While the present model does reproduce the hump-shape of consumption reported in the data and by the two authors cited above, the size and moment of the peak diverge quite significantly from these results. \\
One possible explanation for this difference may be discount factors of different magnitudes. Note that the calibrations by \cite{FV&K2011} and  \cite{yang2009}, aiming to explain life-cycle profiles, resulted in betas, which were significantly lower. These estimates imply that consumers are a lot less patient compared to my calibration. Impatient consumers tend to consume more in early years and also start to reduce their consumption earlier on, thus leading to an earlier peak and lowering the size of the peak, measured as the difference between initial consumption and the highest consumption. Moreover, this behavior leads to lower life-cycle savings, which would lead to a lower consumption peak, thus further reducing the measured size of the peak.\footnote{See for example \cite{Gourinchas&Parker2002} or \cite{cagetti2003} for a discussion of the sensitivity of the life-cycle profiles with regard to the discount factor. Note that I use the same value for the risk aversion parameter as \cite{yang2009}. \cite{FV&K2011} use a value of comparable magnitude. The differences should therefore not arise due to the value of the $\sigma$.}


\subsection{Portfolio Composition over the Life-Cycle}
As discussed in the previous section, average non-durable consumption over the life-cycle is influenced by the capacity to accumulate assets as well as the choice between durables and liquid assets. The present section discusses the average portfolio composition over the life-cycle. 

\begin{figure}[!htbp]
\caption{Average Asset Holdings over the Life-Cycle} 
\label{asset_holdings_life_cycle}	%label, um spaeter auf die Graphiknummer zugreifen zu koennen
\centering
\includegraphics[scale=.4]{Figures2/life_cycle_assets}  % width legt Breite der Graphik fest

\begin{minipage}{0.8\linewidth}
\footnotesize{This figure displays the averages over the simulated life-cycle profiles. The blue line displays average net-worth holdings over the life-cycle. The green line shows the average liquid asset holdings over the life-cycle and the purple line the average of durable holdings over the life-cycle.}
\end{minipage}

\end{figure}

\myparagraph{Wealth Portfolio} All three curves in Figure \ref{asset_holdings_life_cycle} display humps. Liquid assets and net-worth peak at age 66, the period when consumers retire and durables peak at 73, exhibiting a later and somewhat slower decline. 

Early on in life households borrow as much as possible to accumulate durables, which leads to a sharp rise in the average durable stock early on in life. Savings in liquid assets hence start out to be negative and then rise as consumers start to save up for retirement. As durables can also be used to insure against idiosyncratic shocks, liquid assets are primarily used to insure against life-cycle risk. Moreover, assets only become relatively more interesting as
the average stock of durables rises. This results from the fact, that their marginal utility, the return on durables, shrinks relative to the interest rate, which is constant over a consumer's life-cycle.

\myparagraph{What does the Literature say?} The model exhibits a similar pattern of change in portfolio composition described in Section \ref{facts}. The replication of portfolio composition up to retirement, does indeed correspond to the pattern reported by \cite{FV&K2011}. Until the 40ies, the total value of average durable holdings exceeds the value of the average net-worth. After that the former's importance decreases substantially, but does never fall below 50\% in terms of average net-worth. This pattern does also correspond to the age profile of wealth composition displayed Figure 5 in \cite{yang2009} for the portfolio composition of homeowners. \\ However, there are also points where the simulated life-cycle pattern diverges from the data. This is mainly the case for the post-retirement period. In the model, the agents start to run down their assets accumulated during their working-period, whereas in the data average asset holdings stop growing at around retirement and then stay constant.\footnote{See Figure 5 in \cite{yang2009}} Moreover, while the peak of net-worth and liquid assets occurs at around 70 and therefore is quite accurately reproduced by the model, the peak of durables similarly to non-housing occurs much earlier in the data than in the model. \cite{FV&K2011} find a peak in the late 40s and \cite{yang2009} a peak at 55. \\
While the models of these authors do perform better in modeling the durable consumption peak, they also fail to reproduce the stagnation of the asset accumulation later in life.\footnote{The model \cite{yang2009} uses does perform slightly better in modelling the slow downsizing of housing. He shows that including transaction costs as an additional market imperfection leads to a slower decline of housing consumption later in life. However, his model still predicts a strong decline in liquid asset holdings late in life.} The former point may again be due to differences in the discount factor \--- Less patient consumers consume more early in life. The latter observation is mainly due to missing savings motives late on in life. \cite{de2004wealth} shows that adding a bequest motive does improve the model's prediction in this regard. 

\section{What Features of the Wealth Distribution are Explained by the Model?}
\label{distribution_model}
The model does perform quite well in reproducing the life-cycle profiles observed in the data. More so, for the prime age population, which is less concerned with savings motives after retirement. This section discusses the central question, which investigates the models' ability to match the wealth distributions observed in the data.\\
Firstly, I will briefly discuss how I construct a cross-section from the simulated life-cycle profiles and then turn to the discussion of the main results, the predictions for the net-worth distribution up to the 90th percentile. Further, I will show how well the model performs for the relative degrees of inequality. 

\subsection{Creating a Cross-Section from Simulated Life-Cycle Profiles}
In order to proceed with the distributional analysis and to compare the model distribution to the empirical distribution, the above life-cycle profiles have to be converted into a cross-section. In order to do so, I follow \cite{hintermaier2011}, who reproduce the age-composition of the relevant data sample by applying the relevant SCF weights, and then account for cohort effects resulting from income growth. The latter is achieved by reversing the correction for average income growth used, when calibrating the income profiles. The life-cycle model unit output is thus divided by the growth factor $1.015^{(age-base\ age)}$. This ensures that the output is shrunk for cohorts that are relatively older at the time of survey.\footnote{I hereby use the matlab function, compose\_survey.m, provided by \cite{hintermaier2016}.}

\subsection{Net-Worth Distribution up to the 90th Percentile}

\myparagraph{The Results}
Table \ref{ginis_base} displays the distribution means and gini-indices up to the 90th percentile obtained from the data and simulated by the model. The SCF-data moments correspond to the ones displayed in Table \ref{facts_changes}, whereas the model indices and averages are calculated from the net-wealth distribution resulting from the model simulation.\footnote{I hereby use the conventional gini (see Appendix \ref{gini_defs}), used to calculate the data ginis. The resulting bias from negative net-worth values \citep{chen1982} is negligible, as only very net-worth positions are negative.} The means of the two younger age-groups are almost perfectly matched by the model, whereas the average net-worth holdings of the oldest age-group is underpredicted by the model. The simulation results in a lower within-group inequality for the youngest age-group, however, quite accurately matches the concentration of wealth, when it comes to the older two age-groups. On the overall, the model can reproduce the increasing average net-worth holdings and the decreasing concentration of net-worth across the different age-groups observed in the data. 

\begin{table}[!htbp]
\centering
\caption{Means and Ginis of Net-Worth Model vs Data}
\label{ginis_base}
\begin{tabular}{@{}lll@{}}
\toprule
                                                                                & \ \ \ \ \ \ \ \ \ Data                                                 & \ \ \  \ \ \ Model                                                   \\ \midrule
\begin{tabular}[c]{@{}l@{}}Age 26-35 \\ Mean\\ Gini coefficient\end{tabular}  & \begin{tabular}[c]{@{}l@{}} \\ \ \ \ \ \ \ \ \ \ \ 0.80\\ \ \ \ \ \ \ \ \ \ 0.713\end{tabular} & \begin{tabular}[c]{@{}l@{}}\\ \ \ \ \ \ \ \ 0.81\\ \ \ \  \ \ \ 0.627\end{tabular} \\ \midrule
\begin{tabular}[c]{@{}l@{}}Age 36-45 \\ Mean\\ Gini coefficient\end{tabular}  & \begin{tabular}[c]{@{}l@{}}\\ \ \ \  \ \ \ \ \ \ \ 2.36\\ \ \ \ \ \ \ \ \ \  0.596\end{tabular} & \begin{tabular}[c]{@{}l@{}}\\ \ \ \  \ \ \ \ 2.36\\ \ \ \ \ \ \  0.585\end{tabular} \\ \midrule
\begin{tabular}[c]{@{}l@{}}Age 46-55 \\ Mean \\ Gini coefficient\end{tabular} & \begin{tabular}[c]{@{}l@{}}\\ \ \ \  \ \ \  \ \ \ \ 4.81\\ \ \ \ \ \ \ \ \ \  0.564\end{tabular} & \begin{tabular}[c]{@{}l@{}}\\ \ \ \  \ \ \ \ 4.28\\ \ \ \ \ \ \  0.546\end{tabular} \\ \bottomrule
\multicolumn{3}{l}{%
  \begin{minipage}{8.5cm}%
    \small Ginis coefficients and average net-worth holdings by age group for the full net-worth distribution and up to the 90th percentile of the net-worth distribution of 2004 U.S data and resulting from the simulated net-worth distribution. 
  \end{minipage}%
}\\
\end{tabular}
\end{table}

Figure \ref{wealth_distr_base} shows the detailed net-wealth distribution for three different age groups up to the 90th percentile. The blue represents the SCF-Data from the year 2004 and the dotted red line shows the distribution resulting from the model simulation. The detailed representation confirms the output presented in Table \ref{ginis_base} and gives more insights regarding the ability of the model to reproduce the observed pattern found in the data.
The plotted graph shows that the distributions from the two younger age groups are matched quite precisely up the 90th percentile. Moreover, the shape of the oldest age group is matched reasonably well for the first 60 percentiles and then deviates, underpredicting the amount of net-worth by the upper percentiles, thus confirming the lower average holdings reported in Table \ref{ginis_base}. There is one second aspect of the data, that the model does not capture well. Namely, in the empirical distributions the poorest agents hold negative net-worth. This fact is most pronounced for the youngest age-group, where agents up to the 10th percentile hold negative net-worth and then decreases for older consumers, whereas the fraction of 46 to 55 years old holding negative net-worth is almost zero in the data. This explains why the gini-index for the youngest age-group predicted by the model is much lower than the one found in the data and indices of the older age groups are matched more closely.\footnote{Note, however, that one has to be cautious, when looking at the poorest consumers. The measurement of lowest net worth percentiles is very inaccurate, since there are only few observations for the lowest net worth percentiles \citep{hintermaier2011}.} 
\begin{figure}[!htbp]
\caption{Net-Worth distribution up to the 90th percentile} 
\label{wealth_distr_base}	%label, um spaeter auf die Graphiknummer zugreifen zu koennen
\centering
\includegraphics[scale=.35]{Figures2/distribution}  % width legt Breite der Graphik fest

\begin{minipage}{0.8\linewidth}
\footnotesize{The figure compares the simulated distribution to the empirical distribution for all age-groups. The blue line represents 2004 SCF-data and the red-dotted line is the distribution produced by the model. Both distributions are plotted up to the 90th percentile.}
\end{minipage}

\end{figure}

The inability of the model to capture the large negative net-worth holdings is due to the specification of the collateral constraint. The model only allows for very limited amounts of negative net-worth holdings, since borrowing has to be collateralized. The fraction, which can be collateralized by income is very close to zero and therefore short positions in liquid assets are countered by durable holdings. Resulting net-worth positions are thus almost always positive.\footnote{As discussed above the minimum of net-worth is given by $-\gamma\underline{y}$, which is implied by the durable constraint set to zero and the collateral constraint. The calibration above implies that $-\gamma\underline{y} = 0.0179$ and thus is larger than zero.} 

\subsection{Relative Degrees of Inequality}
So far I have only discussed net-worth in the distribution section. However, the model does also generate a distribution of consumption as well as different distributions for liquid assets and durables. Other authors have shown that infinite-horizon models with durables do manage to reproduce the relative ranking of inequalities observed in the data (see for example \cite{hintermaier2010} or \cite{diaz2010}). I find similar results for the present life-cycle model specification.


\begin{table}[!htbp]
\centering
\caption{Gini Indices of Different Distributions}
\label{Gini_Ranking}
\begin{tabular}{@{}lllll@{}}
\toprule
       & \ \ Income & \ \ Durables & Liquid Assets & Net Worth \\ \midrule
Model      &  \ \ \ 0.425 & \ \ \ \ 0.36   & \ \ \ \ 0.95        & \ \ 0.66    \\ \midrule
Data        & \ \ \ 0.427  & \ \ \ \ 0.67     & \ \ \ \ 0.97          & \ \ 0.81      \\ \bottomrule
\multicolumn{5}{l}{%
  \begin{minipage}{10.5cm}%
    \small Ginis coefficients for durables, liquid assets and net-worth as well as income for the full distribution. The empirical indices for durables, liquid assets and net worth are taken from \cite{hintermaier2010} and the index for income from \cite{hintermaier2011}, and are based on calculations from SCF 2004 data. The gini coefficient for liquid assets is calculated with the normalized gini (see Appendix \ref{gini_defs}) based on \cite{chen1982} due to negative observations. 
  \end{minipage}%
}\\
\end{tabular}
\end{table}

Table \ref{Gini_Ranking} displays the gini indices for the whole distribution produced by the model and the one found in the data.\footnote{The empirical indices for durables, liquid assets and net worth are taken from \cite{hintermaier2010} and the index for income from \cite{hintermaier2011}.} The model manages to match the ordering for the different inequalities considering the types of the assets as well as net-worth. However, it does not manage to show that durables are more concentrated than income. \cite{diaz2010} show in a model with housing, that this may be overcome by explicitly modeling a rental market.\footnote{As a consequence of rental markets, wealth poor households will decide to rent instead of owning, thus satisfying their durable consumption needs without being able to benefit from the collateral value of the durable object. As the rented object does not account for the housing part of the durable holdings, wealth poor will hold less durables and thus the gini-index of durables increases. Moreover, \cite{diaz2010} show that the other indices are only marginally affected.} \\
Furthermore, comparing the ginis of the net-worth distribution, one can see that the model does not reproduce the large inequality observed in the data. This should not be surprising given the discussion above. Even more so, since the model was calibrate to match the net-worth distribution up to the 90th percentile, which might further worsen the fit for the overall distribution.  


\section{A Counterfactual Experiment}
\label{counterfactual}
The previous discussion has shown that the model is able to reproduce a number of features of the empirical net-worth distribution. In this section I perform a counterfactual experiment to further develop intuition about the role of the collateral constraint and to evaluate the relative impact of the loan-to-value ratio. In order to do so, I simulate the model setting the collateral value of durables to zero leaving the other parameters as in the baseline case. As a consequence the collateral value of durables disappears.\\
It is particularly interesting to look at different levels of the LTV as the frequency of high-leverage loans has increased significantly over the past decades. \cite{pinto2010government} shows that among purchased loans insured by the Federal Housing Administration or purchased by government-sponsored entreprises, the fraction of originations with cumulative leverage excess of 97 percent of the home values was under 5 percent in 1990 but rose to almost 40 percent in 2007. \cite{bokhari2013did} point a similar picture looking at single-family home loans purchased by Fannie Mae. These authors show that mortgages during 1986 to 1992 were skewed towards LTV's below 80\% where only few LTV's where close to 100\%. Comparing the LTV distribution to a panel for the 1983 to 2007 period, they find that the left tail shrunk, with the effect that ratios of 80\% and 100\% became more apparent. Moreover, \cite{cho2012accounting} show that different levels of the loan-to-value ratio  may account for differences in the wealth distributions between the US and Korea.\\
In order to assess the impact of the LTV, I will compare the results of the counterfactual simulation to the baseline results discussed above. The baseline case has an LTV of almost 100\%, whereas in the counterfactual case it is zero. I first illustrate how the LTV affects the optimal consumer decision, then look at the average life-cycle profile and continue with discussing the distributional consequences for the three age groups. Finally, I also show how the different distributions of the endogenous variables are affected. 

\subsection{On the Individual Level}

Figure \ref{policy_downpayment0_age36} displays the policy functions for a 36 year old agent at three different income levels: the lowest income, blue lines, the intermediate income, green lines, and the highest income, red lines.\footnote{Note that the pattern displayed in Figure \ref{policy_downpayment0_age36} is representative for the other ages of the prime age population.} The solid lines represent the baseline policies and the dashed line the policies where durables do not have collateral value.\\ 
The pattern of these lines follows the one described in \cite{hintermaier2010}. Higher labor income leads to higher optimal durable and non-durable consumption. Moreover, the opposite relation between liquid assets and income levels comes down to the persistence of the income shocks, which leads to higher expected future income of high earners today. These may therefore convert more of their current income in current consumption. Further, the collateral constraint is visible in the kinks of the policy functions. Collateral constraint consumers borrow as much as possible against additional durable collateral holdings. Financial assets are thus decreasing in net-worth. Consumers with higher net-worth holdings are not collateral constraint and chose to increase liquid assets in net-worth in addition to increases in the other endogenous variables.
\begin{figure}[!htbp]
\caption{Individual Policies for different LTVs} 
\label{policy_downpayment0_age36}	%label, um spaeter auf die Graphiknummer zugreifen zu koennen
\centering
\includegraphics[scale=.4]{Figures2/policies_36}  % width legt Breite der Graphik fest

\begin{minipage}{0.8\linewidth}
\footnotesize{The figure displays the different policy functions for the baseline case, solid-lines vs. the case with a reduced loan-to-value ratio, dashed-lines of 36 years old. The different colors display different wage levels. The blue lines illustrate the highest income, the green lines are the intermediate income level and the orange/red lines the lowest income level. Thus income states, 21,11 and 1 respectively. Note, for illustrative purposes I excluded the other income states.}
\end{minipage}

\end{figure}
There are several interesting observations to be made with regard to the level of the LTV. Poor agents with higher income are affected more, while consumers with the lowest income are not affected.
Low income households cannot consume much out of their current income as expected future income is low and thus do not accumulate debt in either case. The tighter collateral constraint, however, does restrict consumers with higher earnings in their ability to borrow liquid assets. These remain borrowing constraint for higher levels of net-worth and do accumulate less durables. Further, a tighter constraint implies higher net-worth holdings for the collateral constraint. As the consumers can accumulate less durables when their borrowing is limited, they have less insurance from durables. As a consequence they optimally choose a higher ratio of durable holdings to negative positions in liquid assets, which leads to a higher net-worth and lower non-durable consumption for a given level of current net-worth.

\subsection{Over the Life-Cycle}
Figure \ref{downpayment_vs_baseline_lc} shows how different levels of LTV do translate into differences in the average life-cycle profiles. The red line is the baseline case and the dotted blue line is the case where durables do not have any collateral value. \begin{figure}[!htbp]
\caption{Average Life-Cycle Profiles Different LTVs} 
\label{downpayment_vs_baseline_lc}	%label, um spaeter auf die Graphiknummer zugreifen zu koennen
\centering
\includegraphics[scale=.4]{Figures2/life_cycle_LTV}  % width legt Breite der Graphik fest

\begin{minipage}{0.8\linewidth}
\footnotesize{This figure displays the averages over the simulated life-cycle profiles. The red line displays the baseline case, where the LTV is set to 0.97. The dotted blue line shows the profiles for an LTV of 0.}
\end{minipage}

\end{figure}A lower loan-to-value ratio forces agents to  postpone the accumulation of durables for a few years. As in this case they cannot finance durables with short positions in liquid assets, they cannot immediately increase the durable stock but need to accumulate it step by step. Moreover, agents now hold positive amounts of liquid assets even when they are young, as they are used as main source of insurance, when durables do not have collateral value. The resulting net-worth positions are higher with a lower LTV over the whole life-cycle. As savings is higher, the peak in consumption is higher and appears later in life.
The initial increase of liquid assets and decrease of durables is due to initial conditions. Finally, later in life agents hit the borrowing constraint to consume as much as possible. When restricted in borrowing agents cut back significantly more on durables later in life, as durables cannot be used as collateral to increase savings as much as possible. The large decumulation of durables leads to big increase of consumption late in life. 


\subsection{The Distribution}
Table \ref{ginis_means_LTV} displays the simulated gini and means of the baseline calibration and the counterfactual experiment. A first observation is, that a lower collateral value of durables implies higher net-worth means, which corresponds to the observations made above. The relative rise is most pronounced for the youngest age group and then decreases for the older age groups. 
Moreover, one can observe that all of the ginis drop. Further, the differences across the age groups is much smaller for an LTV of zero meaning that the gini drops more for the younger age groups.


\begin{table}[!htbp]
\centering
\caption{Means and Ginis for Different LTVs}
\label{ginis_means_LTV}
\begin{tabular}{@{}lllll@{}}
\toprule
                                                                                 & Baseline($\mu = 0.97$) & $\mu = 0$
                                                                 
                                                                                \\ \midrule
\begin{tabular}[c]{@{}l@{}}Age 26-35 \\ Mean\\ Gini coefficient\end{tabular}  &  \begin{tabular}[c]{@{}l@{}}\\ \ \ \ \ \  0.81\\ \ \ \ \ \ 0.627\end{tabular} & \begin{tabular}[c]{@{}l@{}}\\ \ \ 1.31\\ \ \ 0.459\end{tabular} \\ \midrule
\begin{tabular}[c]{@{}l@{}}Age 36-45 \\ Mean\\ Gini coefficient\end{tabular}  &  \begin{tabular}[c]{@{}l@{}}\\ \ \ \ \ \  2.36\\ \ \ \ \ \ 0.585\end{tabular} & \begin{tabular}[c]{@{}l@{}}\\ \ \ 3.27\\ \ \ 0.418\end{tabular} \\ \midrule
\begin{tabular}[c]{@{}l@{}}Age 46-55 \\ Mean \\ Gini coefficient\end{tabular} & \begin{tabular}[c]{@{}l@{}}\\ \ \ \ \ \  4.28\\ \ \ \ \ \ 0.546\end{tabular}  & \begin{tabular}[c]{@{}l@{}}\\ \  \ 5.28\\ \ \ 0.432\end{tabular}  \\ \bottomrule
\multicolumn{5}{l}{%
  \begin{minipage}{9.5cm}%
    \small This table shows the gini coefficients and means for the distribution of different age groups up to and with the 90th percentile of the net-worth distribution for the baseline simulation and the simulation with an LTV of 0. 
  \end{minipage}%
}\\
\end{tabular}
\end{table}

These observations are in line with the discussion above. A tighter borrowing constraint affects the poor, who are most apparent in the youngest age group and thus the younger consumers are more exposed to the LTV. Figure \ref{downpayment_vs_baseline} illustrates this well, while the whole distribution up to the 90th percentile is affected in the youngest age group, only the lower percentiles are affected for the older age groups. It shows the distributions for the two different levels of the LTV. The red line represents the results obtained from the baseline calibration and the dotted blue line the steady state distribution for the prime age sample when durables do not have collateral value. 

\begin{figure}[!htbp]
\caption{Net-Worth Distribution up to the 90th Percentile for Different LTVs} 
\label{downpayment_vs_baseline}	%label, um spaeter auf die Graphiknummer zugreifen zu koennen
\centering
\includegraphics[scale=.4]{Figures2/distribution_LTV}  % width legt Breite der Graphik fest

\begin{minipage}{0.8\linewidth}
\footnotesize{The figure compares the simulated distributions for different levels of the LTV. The red line represents the baseline case, where the LTV is 0.97 and the dotted blue line is distribution resulting from the simulation with an LTV of 0.}
\end{minipage}

\end{figure}

\subsection{Inequality of Different Distributions}
Table \ref{Gini_Ranking_counter_factual} displays the gini indices for the distribution of the endogenous variables. Not surprisingly, both the concentration of liquid assets and net-worth is lower with a lower LTV. Poor consumers hold higher positions in both, which implies a lower inequality. Moreover, the holdings of durables are more concentrated, since the higher liquid asset holdings of the borrowing constraint go hand in hand with lower durable holdings. Finally, consumption is not affected much.  

\begin{table}[!htbp]
\centering
\caption{Inequality of Different Distributions LTVs}
\label{Gini_Ranking_counter_factual}
\begin{tabular}{@{}lllll@{}}
\toprule
      & Consumption & Durables & Liquid Assets & Net Worth \\ \midrule
Baseline & 0.349       & 0.361   & 0.951        & 0.662    \\ \midrule
Counterfactual  & 0.341         & 0.422     & 0.796          & 0.574      \\ \bottomrule
\multicolumn{5}{l}{%
  \begin{minipage}{12.5cm}%
    \small Ginis coefficients for durables, liquid assets and net-worth as well as consumption for the full simulated distribution for different levels of the LTV. The gini coefficient for liquid assets is calculated with the normalized gini (see Appendix \ref{gini_defs}) based on \cite{chen1982} due to negative observations. 
  \end{minipage}%
}\\
\end{tabular}
\end{table}


\section{Conclusion}
\label{conclusion}
The life-cycle model with durables and endogenous borrowing constraints calibrated to match micro-level household data in the US in 2004 is able to match a number of features in the data. The model is successful in reproducing important features of the wealth distribution for consumers of ages 26 to 55 and with net-worth holdings up to the 90th percentile of the net-worth distribution. It correctly predicts that average net-worth holdings increase in age, while net-worth inequality decreases in age. Moreover, it matches the wealth distribution for the working age population between 26 and 45 quite accurately. Furthermore, the model reproduces the observed shapes of the average consumption life-cycle profile and the average portfolio composition up to retirement. The endogenous borrowing constraint with durables serving as collateral helps to explain the surge in durable consumption early in life and the increase of liquid assets later in life. Finally, the model can also predict the relative ranking of inequalities, with respect to durables, liquid assets and net-worth. \\
As is to be expected from such a parsimonious model, it has some limits. Firstly, it underpredicts the accumulation of wealth of the oldest age group and secondly, the model is not able to capture substantial negative net-worth positions of households early in life. Enriching the model with features such as for example, additional savings motives later in live \citep{de2004wealth} or modeling heterogenous rates of returns \citep{benhabib2011distribution} may improve the models prediction in this regard. The relatively higher net-worth positions of young consumers in the model compared to the data, comes down to the borrowing constraint. As the counterfactual experiment shows, these consumers may be strongly affected by the specification of the borrowing constraint. The assumption of full collateralization is to rigid in this respect. \cite{FV&K2011} allow for relatively more borrowing. In their model, households to borrow up to the point at which, for all possible realizations of next period's labor-income shock, consumers  have an incentive to repay their debt rather than to default. When defaulting, consumers would loose all their debt, but also their consumer durables.


%_________________ ENDE DES HAUPTTEILS_________________%


\newpage


%_________________ Literaturverzeichnis _______________%

\addcontentsline{toc}{section}{References}        % Fuegt im Inhaltsverzeichnis "References" hinzu
\bibliographystyle{econometrica_my_version}
\bibliography{bib_thesis}                         % Erstellt Literaturverzeichnis (bindet das file bib_thesis.bib ein




%_________________ Appendix _______________%
\newpage\pagenumbering{Roman}
\appendix
\section{Experience Premium}
 \label{table_experience_premium}  

% Please add the following required packages to your document preamble:

\begin{table}[!htbp]
\centering

% \def\arraystretch{2}
%\setlength{\tabcolsep}{6em}
\caption{Annualized Experience Premium of Labor Income}
\label{my-label}
\begin{tabular}{@{}llll@{}}
\toprule
 \ \ \ Experience \ \ & \ \ 10 years & \ \ 20 years & \ \ 30 years  \\ \midrule
  \ \ \ Premium \ \ & \ \ \ 2.60\%  & \ \ \ 2.18\%  & \ \ \ 1.83\% \\
\bottomrule
\multicolumn{4}{l}{%
  \begin{minipage}{9.5cm}%
\small The annualized experience premium of labor income at ages 36, 46 and 56 relative to age 26. Source: Corresponds to Table 4 in \cite{hintermaier2011} and displays the results of the authors' calculation based on SCF 2004 Data.
  \end{minipage}%
}\\
\end{tabular}
\end{table}

\section{Calibration Method}
\label{estimation_procedure}
In order to calibrate the discount factor and the non-durable consumption weight I follow the simulated methods of moments (SMM) approach. I simulate the model for different combinations of economical plausible values of these parameters looking for the pair that produces simulated moments, which best match these same moments observed in the data.\footnote{Note, since I only look at the wealth distribution up to the 90th percentile, the calibrated parameters do not suffer from the bias put forward by \cite{cagetti2003}. This author showed that estimating preference parameters by matching means results in a bias due to the extreme concentration of wealth at the right tail of the distribution.} I thus look for a vector $\hat{\Theta}$ containing the optimal parameter values $\hat{\beta}$ and $\hat{\theta}$ such that: 

\[ \widehat{\Theta} = \operatorname{arg\,min}[(m(\Theta )-\nu)'I(m(\Theta )-\nu)] \],

where $\Theta$ is a vector of vector containing $\beta$ and $\theta$, $m$ contains the simulated moments for a specific $\Theta$ and $\nu$ is the empirical counterpart. 
Since I follow an exact identified strategy, the simulated moments are weighted by the identity matrix. 

\section{Data}
\label{data}
The data appendix describes how I construct the data counterparts for the initial conditions as well as for the empirical moments used to calibrate the model. This data, unlike the data used for the labor-earnings as well as the net-worth distributions of the different age groups, is not directly taken from \cite{hintermaier2011}. For a detailed description of the latter, I refer to their paper. \\
In order to obtain the initial conditions and the average net-worth holdings and average durable holdings of the prime age population up to the 90th percentile of the net-worth distribution, I used the following definitions for durables, net financial assets and net worth (These closely follow the definitions in \cite{hintermaier2010} and \cite{hintermaier2011}.). \\
\textbf{Net financial assets} are defined as the sum of gross financial assets and total debt. Whereas, \textbf{Gross financial assets} are defined as the sum of money in checking accounts, savings accounts, money-market accounts, money-market mutual funds, call accounts in brokerages, certificates of deposit, bonds, account-type pension plans, thrift accounts, the current value of life insurance, savings in bonds, other managed funds, other financial assets, stocks, mutual funds, owned non-financial business assets and jewelry, antiques or small durables.
\textbf{Total debt} is defined as the sum of mortgage and housing debt, other lines of credit, debt on residential and non-residential property, debt on non-financial business assets, credit-card debt, installment loans, pension loans and margin loans.\\
\textbf{Durables} are the sum of the value of homes, residential and non-residential property and vehicles. Durables are thus defined as in \cite{hintermaier2010} and composed of the most important durable items that can be used as collateral in debt contracts. \\  \textbf{Net-worth} is defined as the sum of net financial wealth and durables.  \\
The sample selection criteria follows \cite{hintermaier2011}. To contain the effect of outliers on the means, observations are dropped if gross labor is negative or net-worth is smaller than -1.2 in terms of the population average of disposable labor income. Moreover, the sample is restricted to households with a household head between age 20 and 55. 
Finally, the statistics reported are based on a pooled sample, where the weights are divided by a factor of 5 so that the weights again add up to the total population size. \\
In order to obtain the data-parts, which were not readily available in the software components of \cite{hintermaier2011}, I used the Stata-Codes provided by the same authors in \cite{hintermaier2016} and modified these codes to account for the definition and sample selection discussed above. In order to construct disposable labor-earnings for each household in the SCF 2004, I use the programs by Kevin Moore provided on $http://www.nber.org/~taxsim/$.  I directly feed the Stata- Datafile $http://users.nber.org/~taxsim/to-taxsim/scf/dta/$ provided by Kevin Moore into the taxsim9 program. Proceeding in this way does result in initial conditions with net-worth holdings that closely match these of \cite{hintermaier2011}, with the additional benefit that the former show how much 23 to 25 years old hold in durables and how much they hold in financial assets. The initial conditions are made available with the software components accompanying the thesis.


\section{Initial Conditions}
\label{initial_conditions}

I re-estimated the model setting the initial conditions from \cite{hintermaier2011} in net-worth equal to durable holdings and liquid assets to zero, when simulating the life-cycle profiles. I repeated this process setting net-worth equal to liquid assets and initial durable holdings to zero.\\
Table \ref{estimates_initial_cond} depicts the estimation results for the preference paramters $\theta$ and $\beta$ as well as the estimated moments. The table shows that the moments are matched up to a precision $10^{-2}$ for both cases. 

\begin{table}[!htbp]
\centering
\caption{Calibration Results for Different Initial Conditions}
\label{estimates_initial_cond}
\begin{tabular}{llllll}
\hline
\multicolumn{2}{l}{Init. Cond} & \ \ \ \ $\beta$ & \ \ \ \ $\theta$ & Av.Durables & Av.Net-Worth\\ \hline
\multicolumn{2}{l}{Durables}             & \ \ 0.991    & \ \ 0.763  & \ \ \ \ 2.9465 & \ \ \ \ 2.3910      \\
\multicolumn{2}{l}{Assets}            & \ \ 0.994   & \ \ 0.752   & \ \ \ \ 2.9530 & \ \ \ \ 2.3875\\ \bottomrule
\multicolumn{6}{l}{%
  \begin{minipage}{11.5cm}%
    \small This table depicts the parameter estimates matching the average durable holdings and average net-worth of the prime age population up to the 90th percentile of the net-worth distribution found in the data for different specifications of the initial conditions. Moreover, the simulated means of net-worth holdings and the durable stock for the prime age population and up to the 90th percentile of the net-worth distribution.
  \end{minipage}%
}\\    
\end{tabular}
\end{table}

Table \ref{ginis_means_init} shows prime age moments for the three different estimations as well as for the data. Attributing the initial net worth entirely to assets produces results that are way off for the youngest age group. The results for the middle age group are matched reasonably well and for the oldest age group they are matched rather badly. Interestingly, when attributing all initial wealth to durables the results are almost the same as for the baseline case. This, however, is not surprising as young households hold almost the entirety of their wealth in durables. Moreover, as discussed above, the specification of the borrowing constraint, does only allow for little negative net-worth. Positive net-worth may therefore well be interpreted as durable holdings. Finally, the assumption, that all positive initial net-worth is durable wealth, would be a sufficient one, for the present model specification. 

\begin{table}[!htbp]
\centering
\caption{Ginis and Means for Different Initial Conditions}
\label{ginis_means_init}
\begin{tabular}{@{}lllll@{}}
\toprule
                                                                                & Data & Baseline & Durables
                                                                 & Assets
                                                                                \\ \midrule
\begin{tabular}[c]{@{}l@{}}Age 26-35 \\ Mean\\ Gini coefficient\end{tabular}  & \begin{tabular}[c]{@{}l@{}} \\0.80\\ 0.713\end{tabular} & \begin{tabular}[c]{@{}l@{}}\\\ \ \ 0.81\\ \ \ \ 0.627\end{tabular} & \begin{tabular}[c]{@{}l@{}}\\ \ \ \ 0.82\\ \ \ \ 0.628\end{tabular} & \begin{tabular}[c]{@{}l@{}}\\0.65\\ 0.669\end{tabular}\\ \midrule
\begin{tabular}[c]{@{}l@{}}Age 36-45 \\ Mean\\ Gini coefficient\end{tabular}  & \begin{tabular}[c]{@{}l@{}}\\ 2.36\\ 0.596\end{tabular} & \begin{tabular}[c]{@{}l@{}}\\ \ \ \ 2.36\\ \ \ \ 0.585\end{tabular} & \begin{tabular}[c]{@{}l@{}}\\ \ \ \ 2.36\\ \ \ \ 0.584\end{tabular} & \begin{tabular}[c]{@{}l@{}}\\ 2.37\\ 0.584\end{tabular} \\ \midrule
\begin{tabular}[c]{@{}l@{}}Age 46-55 \\ Mean \\ Gini coefficient\end{tabular} & \begin{tabular}[c]{@{}l@{}}\\ 4.81\\ 0.564\end{tabular} & \begin{tabular}[c]{@{}l@{}}\\ \ \ \ 4.28\\ \ \ \ 0.546\end{tabular}  & \begin{tabular}[c]{@{}l@{}}\\ \ \ \ 4.28\\ \ \ \ 0.546\end{tabular} & \begin{tabular}[c]{@{}l@{}}\\ 4.50\\ 0.539\end{tabular}\\ \bottomrule
\multicolumn{5}{l}{%
  \begin{minipage}{10.5cm}%
    \small This table shows the gini coefficients and means for the distribution of different age groups up to and with the 90th percentile of the simulated net-worth distribution for calibrations differing in the specification of the initial condition. The first column displays the data moments. Columns 2 to three show the simulated moments for the three different calibrations. The baseline uses initial conditions that are separable in liquid assets and durables. The third Column shows the case, where the initial conditions in net-worth provided by \cite{hintermaier2011} are entirely attributed to durables. Finally, the last column shows the resulting distribution, when these initial conditions in net-worth are attributed to liquid assets. 
  \end{minipage}%
}\\    
\end{tabular}
\end{table}

\begin{figure}[!htbp]
\caption{Distribution with Different Initial Conditions} 
\label{initial_conditions_dist}	%label, um spaeter auf die Graphiknummer zugreifen zu koennen
\centering
\includegraphics[scale=.4]{Figures2/distribution_init_cond}  % width legt Breite der Graphik fest

\begin{minipage}{0.8\linewidth}
\footnotesize{The Figure shows the different simulated net-worth distributions up to the 90th percentile. The blue line shows the baseline distribution with initial conditions that are separable in liquid assets and durables. The red line shows the case, where the initial conditions in net-worth provided by \cite{hintermaier2011} are entirely attributed to durables. Finally, the dotted line shows the resulting distribution, when these initial conditions in net-worth are attributed to liquid assets.}
\end{minipage}

\end{figure}

\newpage

\section{Gini Coefficient}
\label{gini_defs}
This appendix shows the gini specifications, which are used above. For a population uniform on the values $y_{i}, i=1$ to $n$, indexed in non-decreasing order $(y_{i} \leq y_{i+1})$ the gini can be expressed as follows.  
\subsection{Conventional Gini}

\begin{equation*}
G = \frac{1}{n} \left(n+1-2 \left( \frac{\sum_{i=1}^n (n+1-i)y_{i}}{\sum_{i=1}^n y_{i}}\right)\right)
\end{equation*}

and in its simplified version: 

\begin{equation*}
G = \frac{2\sum_{i=1}^n iy_{i}}{n \sum_{i=1}^n y_{i}} - \frac{n+1}{n}
\end{equation*}

\subsection{Normalized Gini}
The normalized Gini proposed by \cite{chen1982}. 

\begin{equation*}
G^{*} = \frac{1+\frac{2}{n}\sum_{i=1}^k iy_{i} - \frac{1}{n}\sum_{k+1}^n y_{i}(1+2(n-i))}{1+\frac{2}{n}\sum_{i=1}^k iy_{i}}
\end{equation*}

where $k$ is defined such that $\sum_{i=1}^k y_{i} = 0$.

\newpage



\section*{Statement of Authorship}
I hereby confirm that the work presented has been performed and interpreted solely by myself except for where I explicitly identified the contrary. I assure that this work has not been presented in any other form for the fulfillment of any other degree or qualification. Ideas taken from other works in letter and in spirit are identified in every single case.
\\[1in] November 15, 2017


\end{document}